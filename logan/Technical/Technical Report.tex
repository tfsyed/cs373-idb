%\documentclass[12pt,letterpaper,onecolumn]{article}
\documentclass[11pt,letterpaper,onecolumn]{article}
%\documentclass[10pt,letterpaper,onecolumn]{article}  % not recommended
%\documentclass[12pt,letterpaper,twocolumn]{article}
%\documentclass[11pt,letterpaper,twocolumn]{article}
%\documentclass[10pt,letterpaper,twocolumn]{article}
%\usepackage{amsmath}
\usepackage{graphics}
%\graphicspath{{path-to-folder-containing-necessary-graphics}{other folder as necessary}}
\usepackage{amssymb}
\usepackage{float}	
\usepackage{graphicx}			% to include EPS figures
\usepackage{xspace}                     	% flexible spaces after symbols
%\usepackage{wrapfig}			% allows to include jpg files
%\usepackage{psfig} 
%\usepackage[pdftex]{graphicx}
\usepackage{epstopdf} 
\textwidth = 6.5 in
\textheight = 8.0 in
\oddsidemargin = 0.20 in
%\evensidemargin = 0.0 in
%\topmargin =-.10 in
%\headheight = 0.0 in
%\headsep = 0.0 in
\usepackage[superscript,biblabel]{cite}
%\parskip = 0.05in
%\clubpenalty=10000
\linespread{2}

%%%%%%%%%% Definitions %%%%%%%%%%%%%%%
\def \webTitle{Elementary}

%%%%%%%%%%%%%%%%%%%%%%%%%%%%%%


\begin{document}
\title{\vspace{-3.0cm}\bf \webTitle - An in depth view of the Periodic Table}
\author{\
{\bf The Elementalists} - Logan Bishop, Colin Murray, Tehreem Sayed, Jose Bigio, Nathan Giles \\*
CS373 - Software Engineering, Spring 2015 \\*
 {\it Department of Computer Science} \\*
}
\date{19 March, 2015}
\maketitle
\newpage

\section*{Introduction}
%Our problem: Most online periodic tables are outdated and do not incorporate much updated content.  Furthermore, they only serve as reference tools.
The periodic table of Elements is an intricate part of teaching Chemistry at the High school and collegiate level.  This can be a daunting task as the periodic table of elements contains a wealth of information for each element, often too much to be properly conveyed through paper medium.  While there are several web accessible versions of the periodic table of elements, our website (\webTitle) seeks to present this wealth of information in a well organized format that would encourage our websites use in the educational setting.

To this end, we approach the use of this information in a database setting as a welcome opportunity to include interactive pages for each chemical, seeking to include modern media such as videos, interactive electron diagrams, and up to date images of the chemicals and their interactions.  In this we find our niche.  A periodic table updated for modern discoveries with electronic media previously unseen on similar sites.

 Given the versatility of hosting our own database, we seek to increase the educational value of our website by offering quizzing options for instructors and fascinating content for students.  In doing so, we can extend the user base of our website past the occasional research and desperate high school student. Rather, we wish to evolve past the idea of the periodic table as a simple reference tool and instead present our \webTitle as learning environment.   

\end{document}